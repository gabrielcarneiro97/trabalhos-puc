%% Modelo copiado da abtex2 sob o nome original: abtex2-modelo-trabalho-academico.tex
%% Todos os direitos reservados de acordo com o copyright do arquivo original.
%% abtex2-modelo-trabalho-academico.tex, v-1.9.6 laurocesar
%% Copyright 2012-2016 by abnTeX2 group at http://www.abntex.net.br/ 

\documentclass[
	% -- opções da classe memoir --
	12pt,				% tamanho da fonte
	openany,
	twoside,			% para impressão em recto e verso. Oposto a oneside
	a4paper,			% tamanho do papel. 
	% -- opções da classe abntex2 --
	%chapter=TITLE,		% títulos de capítulos convertidos em letras maiúsculas
	%section=TITLE,		% títulos de seções convertidos em letras maiúsculas
	%subsection=TITLE,	% títulos de subseções convertidos em letras maiúsculas
	%subsubsection=TITLE,% títulos de subsubseções convertidos em letras maiúsculas
	% -- opções do pacote babel --
	english,			% idioma adicional para hifenização
	brazil				% o último idioma é o principal do documento
	]{abntex2}

% ---
% Pacotes básicos 
% ---
\usepackage{lmodern}			% Usa a fonte Latin Modern			
\usepackage[T1]{fontenc}		% Selecao de codigos de fonte.
\usepackage[utf8]{inputenc}		% Codificacao do documento (conversão automática dos acentos)
\usepackage{indentfirst}		% Indenta o primeiro parágrafo de cada seção.
\usepackage{color}				% Controle das cores
\usepackage{graphicx}			% Inclusão de gráficos
\usepackage{microtype} 			% para melhorias de justificação
% ---

% ---
% Pacotes de citações
% ---
\usepackage[brazilian,hyperpageref]{backref}	 % Paginas com as citações na bibl
\usepackage[alf]{abntex2cite}	% Citações padrão ABNT

% --- 
% CONFIGURAÇÕES DE PACOTES
% --- 

% ---
% Configurações do pacote backref
% Usado sem a opção hyperpageref de backref
\renewcommand{\backrefpagesname}{Citado na(s) página(s):~}
% Texto padrão antes do número das páginas
\renewcommand{\backref}{}
% Define os textos da citação
\renewcommand*{\backrefalt}[4]{
	\ifcase #1 %
		Nenhuma citação no texto.%
	\or
		Citado na página #2.%
	\else
		Citado #1 vezes nas páginas #2.%
	\fi}%
% ---

% ---
% Informações de dados para CAPA e FOLHA DE ROSTO
% ---
\titulo{Observação Participante - Comunidade Cigana}
\autor{Gabriel Carneiro \\ Gustavo Cortez \\ William Lincoln \\ Yan Késsimos}
\local{Brasil, Minas Gerais}
\data{2018}
\orientador{Ângela Maria Siman}
\coorientador{}
\instituicao{%
	Pontifícia Universidade Católica de Minas Gerais - PUC Minas
  \par
  Psicologia - São Gabriel
  \par
  Graduação}
\tipotrabalho{Trabalho Acadêmico}
% O preambulo deve conter o tipo do trabalho, o objetivo, 
% o nome da instituição e a área de concentração 
\preambulo{Trabalho de campo desenvolvido na Comunidade Cigana Calon do São Gabriel - Belo Horizonte/MG.}
% ---


% ---
% Configurações de aparência do PDF final

% alterando o aspecto da cor azul
% \definecolor{blue}{RGB}{41,5,195}

% informações do PDF
\makeatletter
\hypersetup{
     	%pagebackref=true,
		pdftitle={\@title}, 
		pdfauthor={\@author},
    	pdfsubject={\imprimirpreambulo},
	    pdfcreator={LaTeX with abnTeX2},
		pdfkeywords={abnt}{latex}{abntex}{abntex2}{trabalho acadêmico}, 
		colorlinks=true,       		% false: boxed links; true: colored links
    	linkcolor=black,          	% color of internal links
    	citecolor=black,        		% color of links to bibliography
    	filecolor=magenta,      		% color of file links
		urlcolor=black,
		bookmarksdepth=4
}
\makeatother
% --- 

% --- 
% Espaçamentos entre linhas e parágrafos 
% --- 

% O tamanho do parágrafo é dado por:
\setlength{\parindent}{1.3cm}

% Controle do espaçamento entre um parágrafo e outro:
\setlength{\parskip}{0.2cm}  % tente também \onelineskip

% ---
% compila o indice
% ---
\makeindex
% ---

% ----
% Início do documento
% ----
\begin{document}

% Seleciona o idioma do documento (conforme pacotes do babel)
%\selectlanguage{english}
\selectlanguage{brazil}

% Retira espaço extra obsoleto entre as frases.
\frenchspacing

% ----------------------------------------------------------
% ELEMENTOS PRÉ-TEXTUAIS
% ----------------------------------------------------------
\pretextual

% ---
% Capa
% ---
\imprimircapa
% ---

% ---
% Folha de rosto
% (o * indica que haverá a ficha bibliográfica)
% ---
\imprimirfolhaderosto

% ---
% inserir o sumario
% ---
\pdfbookmark[0]{\contentsname}{toc}
\tableofcontents*
\cleardoublepage
% ---

% ----------------------------------------------------------
% ELEMENTOS TEXTUAIS
% ----------------------------------------------------------
\textual

% ---
% Capitulo com exemplos de comandos inseridos de arquivo externo 
% ---
\include{abntex2-modelo-include-comandos}

\chapter{Introdução}
Visitamos a comunidade cigana dos Calons localizada no São Gabriel - Belo Horizonte. 
Quando chegamos a comunidade notamos que as barracas, que esperávamos encontrar em uma 
comunidade cigana não estavam mais no local. Conversando com alguns moradores da região 
descobrimos que a prefeitura em 2013 cedeu o espaço onde os ciganos moravam para eles, dando assim o 
direito de propriedade do espaço \cite{recebem-imovel}, por isso uma grande quantidade de ciganos que moravam no local se 
mudaram, deixando assim uma pequena parcela dos moradores ciganos na região \cite{dilemas-diversidade}.

Tivemos a oportunidade de conversar com alguns ciganos que nos contaram 
sobre suas experiências de vida e sobre seus parentes que já não estavam mais
instalados na região. Os ciganos que ainda moravam ali já haviam sido de certa forma urbanizados,
pois moravam em casas de alvenaria simples e em algumas casas era possível observar a presença de
energia elétrica.

\chapter{Caracteristicas dos sujeitos}
Os entrevistados foram pessoas que tem como estilo de vida a "vida cigana". As pessoas eram ciganos Calon. É importante destacar que os ciganos são divididos
em três outras etnicidades os: Calon, Rom e Sinti \cite{embaixada-cigana}. Como os indivíduos são pertencentes a estes tipo de
ramificação eles possuem características próprias. 

A primeira e mais expressiva é o jeito de se vestir e
falar. Os ciganos, em geral, se comunicam através de uma língua específica do grupo cigano chamada de Romani, contudo os
ciganos da ramificação Calon comunicam-se através do "shib" este por sua vez é um dialeto constituído
pela fusão do idioma local mais algumas expressões do Romani, mas a peculiaridade não
se encontra apenas em termos desconhecidos, mas também com entonações que são diferente do português
brasileiro. As vestimentas como muitas culturas se diferencia de acordo com o sexo e idade, as mulheres
quase sempre utilizam saias grandes ou vestidos coloridos, alguns adornos que na maioria das vezes são
feitas pela própria comunidade cigana, maquiagens vivas como batons e sombras e cabelos compridos e
sem uso de química alguma. É possível observar que o padrão estético de beleza dos
ciganos diferem ao nosso uma vez que a grande mídia não tem efeito relevante sobre eles. Os homens em
detrimento das mulheres possuem uma vestimenta semelhante de um homem do campo, apresentam-se quase
sempre com a cabeça coberta, seja por chapéu ou boné, camisas simples na maioria das vezes de botões, calças
compridas acompanhadas de tênis ou bota.

Foi observado certo padrão na vestimenta tanto das mulheres, quanto dos homens. Mulheres utilizavam, em geral,
saias longas com uma camisa um pouco mais confortável, algumas utilizando casacos de linha. Homens
utilizavam camisas um pouco mais “jogadas”, em geral com calça Jeans e algum tênis ou chinelo. Mulheres
em geral com pelos sem fazer e homens em geral com barbas grandes.

\chapter{O que é ouvido?}
Os ciganos são um povo alegre, e expressam sua alegria em forma de música, mesmo
com o avanço da tecnologia eles dão preferência às músicas tocadas por instrumentos e cantadas pelos
próprios membros da comunidade. Como já dito eles possuem uma peculiaridade no vocabulário que
permite uma interação apenas deles. Em alguns momentos podemos observar erros de concordância nas falas em português,
mas que podemos definir não como analfabetismo ou falta de escolaridade, mas com um artifício que os
ciganos encontraram de definir sua identidade na comunicação. A entrevistada em especial mantém o hábito
de estalar os dedos enquanto falava, este hábito é possível associado a musicalidade interior dos ciganos
que permitem que eles transmitem através de algum gesto o que estão sentindo.

Os ciganos entrevistados tem uma maneira um tanto quanto única de falar. Frases um pouco mais lentas
e um pouco mais do interior, um linguajar não muito sofisticado e com um pouco de resistência ao falar
conosco. Em geral era bem quieto o local, não havia muitas músicas ou diálogos (aparentemente).

\chapter{Descrição do espaço físico}
Em geral os ciganos são nômades e sempre se instalam em regiões
estratégicas, como por exemplo terminal de metrôs e rodovias, isto se dá pelo fato dos ciganos
adquirirem sua renda do comércio, prestação de serviço e entretenimento. As tendas são feitas de lonas e
são montadas em qualquer espaço inocupado, contudo na região que realizamos as entrevistas as tendas já não existiam mais. Para compreendermos essa situação é necessário entender
como surgiu essa comunidade de Ciganos Calon no bairro São Gabriel \cite{dilemas-diversidade}. Os ciganos encontraram uma
região rural muito extensa e se instalaram, tem-se relatos de ciganos que chegaram na região antes mesmo de sua urbanização. Com o tempo alguns ciganos foram embora e outros continuaram na região, com isso a prefeitura legalizou o terreno e cedeu para os ciganos
remanescentes. Muitos ciganos venderam sua terra e continuam sua peregrinação. Este fato contribuiu para que 
surgissem casas em detrimento das tendas, alguns ciganos que ainda residem moram em casas de alvenaria,
a aglomeração tem características de uma pequena favela, com casas pequenas, sem acabamento e saneamento básico. 

Mesmo após passarem por esse processo de urbanização, observamos
que muitos costumes ainda são preservados de modo oral como o dialeto próprio deles o Romani, 
língua dos ciganos é a maior língua não regional ágrafa do mundo.

O local onde fomos já estava ocupado, já haviam lotes dados pela prefeitura e casas compradas pelos ciganos, com algumas tendas e pouquíssimas barracas, a maioria eram casas de tijolo,
em geral muito humildes e com algumas decorações coloridas, roupas extensas nas portas das casas.

\chapter{Comentários, reflexões, opiniões e sentimentos}
As opiniões são várias quanto ao povo cigano, a mais relevante delas é
como se encontra os ciganos hoje, a senhora entrevistada não se sentia mais parte de uma
comunidade cigana, por mais que ela mantivesse a cultura cigana. Algumas tradições
dizem respeito a estrutura social da comunidade a entrevistada conta que na antiga
comunidade cigana existia um líder da comunitário chamado de "baro" que defendia os interesses da
comunidade e à representava diante de qualquer comissão. Era comum também a existência das chamadas
"criss" que são uma espécie de tribunal que julgava e culpava pessoas de determinadas ações, com a
difusão da polícia esses tribunais foram se tornando cada vez mais dispensáveis. É perceptível que o Estado possui uma preocupação com os ciganos, mas não toma medidas realmente efetivas. O
fato de a prefeitura legalizar a situação dos ciganos com suas terras foi um ponto positivo, mas a falta de
mecanismos que mantivessem a cultura cigana, possibilitou o surgimento de um complexo habitacional que
não corresponde com a cultura cigana e suas prerrogativas. Os poucos ciganos restantes, ainda se encontram em
uma situação de pobreza que merece a preocupação do Estado. Com esse cenário é fácil associar a
situação do cigano hoje com a sua história, em 1710, o Romani foi proibido no
Brasil \cite{ciganos-exclusao}, na Segunda Guerra mundial 500.000 ciganos foram mortos apenas por serem ciganos. Antes disso
os ciganos principalmente os Sintis, na Europa eram vistos como um povo amaldiçoados e que atraem
pobreza e miséria, isso reflete no estereótipo criado pelo brasileiro de que os ciganos são um povo, sujo e
criminoso.

Encontramos bastante resistência dos ciganos ao falar conosco, assim que fomos perguntar onde poderíamos
encontrar alguém da cultura cigana, fomos rapidamente encaminhados por uma senhora para um senhor que era cigano.
Este senhor também nos encaminhou rapidamente para outra pessoa e esta pessoa para outra. Após uma breve pesquisada, descobrimos que os
ciganos tem um pouco de receio de falar do passado e talvez tenha sido isto que impactou nossa visita.

\chapter{Detalhamento das atividades}
Chegamos ao local, um acampamento cigano um pouco próximo da PUC São Gabriel, onde demoramos um pouco para
encontrar o lugar onde realmente os ciganos estavam instalados, mas com um pouco de procura
finalmente encontramos. Seguimos a pé por alguns minutos até um pouco mais a fundo do lugar, onde vimos várias
casas, sem muitas pessoas na rua. Perguntamos onde poderíamos encontrar os ciganos em um posto de saúde,
onde fomos informados que eles haviam vendido algumas propriedades deles já faz algum tempo, mas ainda haviam
alguns.

A comunidade foi encontrada pela internet, ao chegar no local ocorreu uma
quebra de expectativa, pois era esperado encontrarmos tendas, músicas e uma quantidade de pessoas muito
grande, contudo em virtude da história da comunidade já relatada foram encontradas casas pequenas com
pouca infraestrutura assemelhando-se a um complexo habitacional materialmente esquálido.

A primeira dificuldade encontrada pelo grupo foi de encontrar ciganos e de estabelecer uma relação que 
proporciona-se segurança ao entrevistado. Os ciganos têm um costume bastante peculiar que é a não glorificação
do passado, para tanto, em algumas vertentes ciganas,
quando um membro morre, suas roupas e pertences pessoais são enterrados ou queimados. Atribuímos de certa forma a esse costume a 
dificuldade que tivemos de conversar com alguns indivíduos, em especial os mais velhos. Como exemplo, não se tem registro de livros sobre ciganos
que foram escritos por próprios ciganos, este fato contribui para muito para a criação de um preconceito já
que quem fala sobre ciganos são pessoas que não são ciganos.

Outro fato que observamos foi o grande número de palavras desconhecidas que foram
utilizadas pelos entrevistados que tinham que ser "traduzidas", essas palavras contam também um
pouco da história e organização dos ciganos.

Em termos gerais os ciganos Calon mantém práticas rurais como meio de subsistência, mas com a
globalização eles passaram a ter atividades cada vez mais urbanizadas como o comércio e a prática do
"draba".

O caráter mais desmistificador foi descobrir que a vida cigana não é uma religião, existem
ciganos de diversas religiões e algumas religiões podem até condenar práticas ciganas, mas ser cigano não
é ter uma conduta de caráter religioso, mas todos ciganos devem acreditar em Deus, segundo a
embaixada cigana \cite{embaixada-cigana} acreditar que ciganos são uma religião é apenas uma das diversas precondições criadas
que afastam as pessoas da verdadeira essência que é ser cigano.


 


\postextual

\bibliography{visita-de-campo-ciganos-referencias}

\end{document}
